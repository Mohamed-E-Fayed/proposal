\documentclass[
	%a4paper, % Use A4 paper size
	letterpaper, % Use US letter paper size
]{jdf}

\addbibresource{references.bib}

\author{Mohamed Essam Moustafa Kamel Fayed, Varun Ramakrishnan, Bruce Walker}
\email{mohamed.fayed@gatech.edu, vramakri6@gatech.edu, bruce.walker@psych.gatech.edu}
\title{Graph Ingestion Engine (Fall 2024)}

\begin{document}
%\lsstyle

\maketitle

\begin{abstract}

    Graph Ingestion Engine (GIE) is a project that aims at reverse engineering graphs into their underlying data points.
    This is important for digitizing those graphs into more compressed form and making charts accessible to people with visual disabilities.
    In this semester, we will continue on previous efforts towards this goal.
\end{abstract}

\section{Introduction}\label{sect:introduction}

%Charts are used to explore and get insights about underlying data.
%Unfortunately, they are embedded into images formats with no meta data about their content, thus providing no information to screen readers and their users.
Charts are convenient methods in conveying messages about underlying data.
However, they are stored as images, thus consuming a lot of storage capacity.
Moreover, they are inaccessible to screen readers, thus not informative to visually impaired people.

There has been growing interest in converting them into accessible formats, by converting them into tables~\cite{liu2022deplot},. 
Another direction is concerned with Chart Question Answering~\cite{masry2022chartqa,masry2024chartgemma}.

%In our work, we focus on 

\section{Related Work}
\label{sect:related}
There are many types of graphs, and each type proposes a different set of challenges in extracting their underlying data, in a process known as deplot.
This necessitates determining the type of input chart.
~\cite{dsouza2023} built a classifier to detect the type of input chart.
~\cite{dsouza2022} explored the usage of image segmentation algorithms to deplot gray scaled scatter plots.
~\cite{dsouza2023} succeeded in making data out of Line Charts and initially good results on Pie Charts.
They showed that Deep Learning is superior to other techniques in Pie Charts.

\section{Scope}\label{sect:scope}

During this semester, we aim at:
\begin{itemize}
         \item  making a deployment that can be tested by external partners,
         \item Assess the quality of models made by other colleagues in the wild, and
         \item Improve the quality of all models in the system.
              \end{itemize}

\paragraph{In details:} 
\begin{enumerate}
        % deployment
    \item Fix connectivity issues between front-end and back-end,
    \item store the input images into disk to be used in training and testing new models,
    \item synchronize the deployment to larger storage,
        % metrics
         \item Make a new metric for evaluating the consistency of models in extracting the trend in data used to generate graphs.
         \item Update the RMS algorithm and handle more input cases,
         \item post-process the output of computer vision models to be compared to ground truth using RMS algorithm,
         \item Continue the survey of testing LLMs in the task of Chart-to-Table.
         \item Continue orgganizing testsets according to difficulty in order to make it more informative in quantifying improvement and track progress, and
         \item Add more types of graphs.
              \end{enumerate}

            
            \section{Timeline}\label{sect:timeline}

            \begin{table}
                 \begin{tabular}{|c|c|c|}
                     Task & Assignee & Due Date \\
                     Convert CV outputs into tables & & \\
                     Fix Deployment & & \\
                     Finish survey of Chart-to-Table & & 02/15 (Deadline for ACL conference) \\
                 \end{tabular}
                 \caption{}
                 \label{}
                  \end{table}

\section{References}
\printbibliography[heading=none]
\end{document}
